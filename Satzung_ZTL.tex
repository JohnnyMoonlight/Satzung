\documentclass[a4paper, 12pt]{scrartcl}

%% PDF SETUP
\usepackage[pdftex, bookmarks, colorlinks, breaklinks,
pdfusetitle,plainpages=false]{hyperref}
\hypersetup{linkcolor=blue,pdfauthor={Zentrum für Technikkultur Landau},citecolor=blue,filecolor=black,urlcolor=blue,plainpages=false} 
\usepackage[utf8]{inputenc}
\usepackage[T1]{fontenc}
\usepackage[ngerman]{babel}

\usepackage{ccicons}
\usepackage{url}
\usepackage{eurosym}

% Optima as a sans serif font.
\renewcommand*\sfdefault{uop}
\usepackage[protrusion=true,expansion=true]{microtype}
% Recalculate page setup based on new font.
\KOMAoptions{DIV=last}
\usepackage{todonotes}
\pagestyle{plain}

\renewcommand*\thesection{\S~\arabic{section}}
\KOMAoptions{toc=flat}

\begin{document}
\title{Satzung}
\subtitle{des Vereins Zentrum für Technikkultur Landau}
\author{}
\date{}

\maketitle

\tableofcontents

\vfill

\begin{flushright}
	\ccby \\
	{\small
		Dieses Dokument steht unter der Creative Commons Namensnennung 3.0 Deutschland Lizenz. Mehr Informationen zur Lizenz unter \url{https://creativecommons.org/licenses/by/3.0/de/}
	}
\end{flushright}

\newpage
\section{Name, Sitz, Geschäftsjahr}
\begin{enumerate}
	\item Der Verein führt den Namen "`Zentrum für Technikkultur Landau"' und soll in das Vereinsregister eingetragen werden; nach der Eintragung führt er den Zusatz "`e.V."'.
	\item Der Verein hat seinen Sitz in Landau in der Pfalz.
	\item Das Geschäftsjahr des Vereins ist das Kalenderjahr.
\end{enumerate}

\section{Zweck des Vereins}
\begin{enumerate}
	\item Zweck des Vereins ist die Förderung der Bildung, Volksbildung und Wissenschaft auf technischen, informationstechnischen und verwandten Gebieten sowie von Kunst und Kultur durch
den kreativen Umgang mit diesen.
	\item Der Verein verfolgt ausschließlich und unmittelbar gemeinnützige Zwecke im Sinne des Abschnittes "`Steuerbegünstigte Zwecke"' der Abgabenordnung.
	\item Der Satzungszweck wird verwirklicht durch Veranstaltungen, Vorträge und Seminare, die vom Verein veranstaltet werden.
\end{enumerate}

\section{Selbstlosigkeit}
\begin{enumerate}
	\item Der Verein ist selbstlos tätig; er verfolgt nicht in erster Linie eigenwirtschaftliche Zwecke.
	\item Mittel des Vereins dürfen nur für die satzungsgemäßen Zwecke verwendet werden. Die Mitglieder erhalten keine Gewinnanteile und in ihrer Eigenschaft als Mitglieder auch keine sonstigen Zuwendungen aus Mitteln des Vereins. Es darf keine Person durch Ausgaben, die dem Zweck des Vereins fremd sind, oder durch unverhältnismäßig hohe Vergütungen begünstigt werden.
	\item Alle Inhabenden von Vereinsämtern sind ehrenamtlich tätig.
\end{enumerate}

\section{Erwerb der Mitgliedschaft}
\label{erwerb-der-mitgliedschaft}
\begin{enumerate}
	\item Mitglieder können natürliche und juristische Personen jedweder Rechtsform werden.
	\item Die Mitgliederversammlung kann Personen, die sich durch besondere Verdienste im Sinne des Vereins oder die von ihm verfolgten satzungsgemäßen Zwecke hervorgetan haben, zu Ehrenmitgliedern ernennen. Ehrenmitglieder haben alle Rechte eines ordentlichen Mitglieds. Sie sind von Beitragsleistungen befreit.
	\item Der Vorstand entscheidet auf schriftlichen Antrag des potentiellen Mitglieds über die Aufnahme. Der Beschluss wird der antragstellenden Person schriftlich oder per E-Mail mitgeteilt.
	\item Gegen den ablehnenden Bescheid des Vorstands kann die antragstellende Person Beschwerde einlegen, die binnen eines Monats ab Zugang der Ablehnung schriftlich beim Vorstand einzureichen ist. Über die Beschwerde entscheidet die Mitgliederversammlung nach demselben Verfahren wie bei Ausschluss eines Mitglieds.
	\item \label{beginn-mitgliedschaft} Die Mitgliedschaft beginnt nach positivem Aufnahmebescheid mit dem Eingang des ersten Mitgliedsbeitrags.
	\item Fördermitglieder sind passive Mitglieder ohne Stimmrecht in der Mitgliederversammlung.
\end{enumerate}

\section{Beendigung der Mitgliedschaft}
Die Mitgliedschaft endet
\begin{enumerate}
	\item bei juristischen Personen mit deren Auflösung.
	\item bei natürlichen Personen mit deren Tod.
	\item nach schriftlicher Austrittserklärung eines Mitglieds, die schriftlich beim Vorstand eingegangen sein muss.
	\item bei Mitgliedern, die sich nach schriftlicher Mahnung mehr als sechs Monate mit Mitgliedsbeiträgen im Verzug befinden.
	\item durch Ausschluss aus dem Verein.
\end{enumerate}

\section{Mitgliedsbeiträge}
\label{mitgliedsbeitraege}
\begin{enumerate}
	\item Der Verein erhebt einen regelmäßigen Mitgliedsbeitrag, der im Voraus zu entrichten ist. Näheres regelt eine von der Mitgliederversammlung zu beschließende Beitragsordnung.
	\item Im Falle nicht fristgerechter Entrichtung der Beiträge ruht die Mitgliedschaft.
	\item \label{mitgliedsbeitraege-mindestbeitrag} Im begründeten Einzelfall kann für ein Mitglied durch Vorstandsbeschluss ein von der Beitragsordnung abweichender Beitrag festgesetzt werden. Aus organisatorischen Gründen, und um den Willen des Mitglieds eindeutig zu erkennen die Mitgliedschaft aufrecht zu erhalten, soll dieser Beitrag minimal 1 \euro/Monat betragen.
	\item Bei Beendigung der Mitgliedschaft, gleich aus welchem Grund, erlöschen alle Ansprüche aus dem Mitgliedsverhältnis. Eine Rückerstattung von Beiträgen, Spenden oder sonstigen Unterstützungsleistungen ist grundsätzlich ausgeschlossen. Der Anspruch des Vereins auf offene Beitragsforderungen bleibt hiervon unberührt.
\end{enumerate}

\section{Organe}
Die Organe des Vereins sind
\begin{enumerate}
	\item Der Vorstand, bestehend aus dem/der:
	\begin{enumerate}
		\item 1. Vorsitzenden
		\item 2. Vorsitzenden
		\item Schatzmeister\_in
		\item bis zu vier Beisitzer\_innen
	\end{enumerate}
	\item die Mitgliederversammlung.
\end{enumerate}

\section{Die Mitgliederversammlung}
\label{die-mv}
\begin{enumerate}
	\item Die Mitgliederversammlung besteht aus den Mitgliedern des Vereins.
	\item Die ordentliche Mitgliederversammlung wird einmal jährlich vom   Vorstand einberufen.
	\item Es können außerordentliche Mitgliederversammlungen entweder auf Beschluss des Vorstands oder auf Verlangen eines Fünftels der Mitglieder einberufen werden.
	\item \label{mv-einladung} Die Einladung zur Mitgliederversammlung ist den Mitgliedern schriftlich oder per E-Mail unter Angabe von Ort, Zeit und Tagesordnung mindestens zwei Wochen vorher zuzustellen. Die Einladung erfolgt an die letzte vom Mitglied bekannt gegebene Adresse.
	\item \label{mv-nachtrag} Mitglieder können zu den bestehenden Tagesordnungspunkten weitere Anträge stellen, wenn sie diese dem Vorstand spätestens eine Woche vor dem anberaumten Termin schriftlich oder per E-Mail zur Bekanntgabe mitteilen. Die Mitgliederversammlung beschließt über die Zulassung der nachträglichen Anträge zur Beschlussfassung.
	\item Eine Vertretung eines Mitglieds durch ein anderes ist möglich, wenn die Vertretungsbefugnis schriftlich nachgewiesen wird.
	\item Jede ordnungsgemäß einberufene Mitgliederversammlung ist
    unabhängig von der Zahl der erschienenen Mitglieder
    beschlussfähig.
	\item Die Mitgliederversammlung wird von einem anwesenden Vorstandsmitglied geleitet. Ist kein Vorstandsmitglied anwesend, bestellt die Mitgliederversammlung einen Versammlungsleiter.
	\item Grundsätzlich ist die Mitgliederversammlung öffentlich. Gäste können auf Beschluss der Versammlung durch einfache Mehrheit ausgeschlossen werden.
	\item{Protokollführung}
		\begin{enumerate}
			\item Die Mitglieder wählen aus ihren Reihen eine protokollführende Person.
			\item Über den Verlauf der Mitgliederversammlungen ist eine Niederschrift anzufertigen, die von der versammlungsleitenden und von der protokollführenden Person zu unterzeichnen ist. Diese Niederschrift ist auf Anfrage beim Vorstand einsehbar. Erfolgt innerhalb von vier Wochen nach Unterzeichnung der Niederschrift kein Einspruch gilt diese als genehmigt.
			\item Die Niederschrift soll folgende Angaben enthalten:
			\begin{enumerate}
	  		\item Ort und Tag der Versammlung
	    	\item Name der versammlungsleitenden sowie der protokollführenden Person
	    	\item die Zahl der erschienen Mitglieder
	    	\item Angaben zu den gefassten Beschlüssen mit genauen Abstimmungsergebnissen
	    	\item die erforderlichen Unterschriften
			\end{enumerate}
		\end{enumerate}
	\item Jedes Mitglied, dessen Mitgliedschaft nicht ruht, ist stimmberechtigt.
	\item Juristische Personen haben eine Stimmberechtigte Person vor der Versammlung schriftlich zu bestellen.
\end{enumerate}

\section{Zuständigkeiten der Mitgliederversammlung}
Die Mitgliederversammlung
\begin{enumerate}
	\item wählt und kontrolliert den Vorstand.
	\item prüft und genehmigt die Jahresabschlussrechnung des Schatzmeisters/der Schatzmeisterin und erteilt die Entlastung des Vorstandes.
	\item entscheidet in allen Fällen, in denen nicht die Zuständigkeit eines anderen Organs bestimmt ist.
	\item trifft Mehrheitsentscheidungen mit der einfachen Mehrheit der teilnehmenden Mitglieder.
	\item kann den Vereinszweck mit der Zustimmung aller teilnehmenden
    Mitglieder ändern. Der Änderungsantrag muss gemäß
    \ref{die-mv}.\ref{mv-einladung} erfolgen. Weiter wird bestimmt, dass
    \ref{die-mv}.\ref{mv-nachtrag} für Zweckänderungen sowie
    Satzungsänderungen keine Anwendung findet. Zweckänderungen und
    Satzungsänderungen können somit nicht durch Nachtrag zur Tagesordnung beschlossen werden.
	\item kann die Vereinssatzung mit Zustimmung von drei Vierteln der teilnehmenden Mitglieder ändern. 
	\item gibt sich eine Geschäftsordnung.
\end{enumerate}

\section{Der Vorstand}
\label{der-vorstand}
\begin{enumerate}
\item Der Vorstand trifft seine Beschlüsse auf Sitzungen, zu denen spätestens eine Woche vorher schriftlich oder per E-Mail zu laden ist. Mit dem Einverständnis aller Vorstandsmitglieder kann diese Frist verkürzt werden oder ganz entfallen.
	\item Der Vorstand ist beschlussfähig, wenn mindestens zwei Vorstandsmitglieder anwesend sind.
	\item Beschlüsse im Vorstand werden mit einfacher Mehrheit gefasst.
	\item Bei Ausscheiden eines Vorstandsmitglieds kann durch den Vorstand für die verbleibende Amtszeit ein Stellvertreter bestellt werden.
	\item \label{der-vorstand-vertretung} Der Verein wird gerichtlich und außergerichtlich (§ 26 BGB) durch die Vorstandsmitglieder vertreten. Der Verein wird durch den/die 1. oder 2. Vorsitzende\_n sowie ein weiteres Vorstandsmitglied nach außen vertreten.
	\item Der Vorstand wird von der Mitgliederversammlung auf die Dauer von zwei Jahren bestellt. Er bleibt jedoch bis zur Bestellung eines neuen Vorstands im Amt. Die Wiederwahl ist zulässig.
	\item Eine juristische Person kann nicht Vorstandsmitglied werden.
	\item \label{der-vorstand-zugehoerigkeit}Vorstandsmitglied kann nur werden, wer mindestens 6 Monate Vereinsmitglied ist und in dieser Zeit für die Ziele des Vereins förderlich
tätig war. Über die Eignung der kandidierenden Person entscheidet die Mitgliederversammlung.
\end{enumerate}

\section{Zuständigkeiten des Vorstands}
\begin{enumerate}
	\item Der Vorstand führt die Geschäfte des Vereins und fasst die erforderlichen Beschlüsse.
	\item Er ist zu rechtsgeschäftlichen Verpflichtungen zu Lasten des Vereins ermächtigt.
	\item Der Vorstand gibt sich eine Geschäftsordnung, die die Aufgabenverteilung innerhalb des Vorstands und die gegenseitige Vertretung der Vorstandsmitglieder, sowie die Art des Zustandekommens seiner Beschlüsse regelt und die der Zustimmung der Mitgliederversammlung bedarf.
\end{enumerate}

\section{Ausschluss von Mitgliedern}
\begin{enumerate}
	\item Der Vorstand kann mit einfacher Mehrheit ein Mitglied auf Antrag ausschließen.
	\item Gegen diesen Ausschluss kann schriftlich Widerspruch eingelegt werden.
	\item Ein Widerspruch führt zu einer Überprüfung des Ausschlusses durch die Mitgliederversammlung. Die einfache Mehrheit kann den Ausschluss ablehnen.
	\item Bis zur Entscheidung der Mitgliederversammlung ruht die Mitgliedschaft.
\end{enumerate}

\section{Auflösung des Vereins}
\label{aufloesung-des-vereins}
\begin{enumerate}
	\item Der Antrag auf Auflösung des Vereins kann durch den Vorstand oder ein Fünftel der Mitglieder gestellt werden.
	\item Die Auflösung des Vereins kann nur in einer eigens zu diesem Zweck einberufenen Mitgliederversammlung mit einer Mehrheit von drei Vierteln der abgegebenen gültigen Stimmen beschlossen werden.
	\item \label{aufloesung-des-vereins-3} Bei Auflösung des Vereins oder bei Wegfall steuerbegünstigter Zwecke fällt das Vermögen des Vereins an eine juristische Person des öffentlichen Rechts oder eine steuerbegünstigte Körperschaft zwecks Verwendung für die Förderung von Wissenschaft und Forschung.
\end{enumerate}

\vspace{2.5cm}

\noindent Durch Beschluss der Mitgliederversammlung vom 16.07.2022 in \ref{mitgliedsbeitraege}.\ref{mitgliedsbeitraege-mindestbeitrag} (Mitgliedsbeiträge) sowie \ref{der-vorstand}.\ref{der-vorstand-zugehoerigkeit} (Der Vorstand) geändert. In Originalfassung beschlossen bei der Gründungsversammlung am 22.01.2019, durch Beschluss vom 13.02.2019 in \ref{der-vorstand}.\ref{der-vorstand-vertretung} (Der Vorstand) und \ref{aufloesung-des-vereins}.\ref{aufloesung-des-vereins-3} (Auflösung des Vereins) geändert.\\[0.5cm]

\noindent Landau, 16.07.2022

\end{document}
